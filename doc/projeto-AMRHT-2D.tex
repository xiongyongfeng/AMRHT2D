\documentclass[12pt, brazil]{article}

\usepackage{amssymb}
\usepackage{amsmath}
\usepackage{subfigure}
\usepackage{tikz}
%\renewcommand{\theequation}{\thechapter.\arabic{equation}}
\usepackage[brazil]{babel}
\usepackage[utf8]{inputenc}
\usepackage{epsfig}
\usepackage{url}
\usepackage{enumitem}
\usepackage{hyphenat}
\usepackage{textcomp}
\usepackage{gensymb}
\usepackage{tikz,pgf}
\usetikzlibrary{arrows}
%\usepackage{slashbox}
\usepackage{pict2e}
\usepackage{wrapfig}
%\usepackage{txfonts}
%\usepackage{pxfonts}
\linespread{1.2} % Line spacing
\textwidth=16.5cm
\textheight=22.5 cm
\topmargin=-.9cm
\evensidemargin=0cm
\oddsidemargin=0cm
\baselineskip=1.5cm
\parskip=0.5pc
\parsep=.7cm

\newcounter{constrcount}
\setcounter{constrcount}{1}
\newcommand{\DOC}{\texttt{[doc\arabic{constrcount}]}\addtocounter{constrcount}{1}}
\newcommand{\bs}{\begin{center}}
\newcommand{\es}{\end{center}}

\newcounter{gambi}
\setcounter{gambi}{1}
\newcommand{\meuCont}{(\arabic{gambi})\addtocounter{gambi}{1}}

\begin{document}

\vspace*{2cm}
\begin{center}
{\large {\bf Projeto AMRHT-2D}} \\

\vspace{5cm}

Descrição dos problemas e métodos estudados





\end{center}
\thispagestyle{empty}

\newpage

\tableofcontents
\thispagestyle{empty}
\newpage
\pagenumbering{arabic}\setcounter{page}{1}


%\vspace{1pc}
%{\bf Simulação numérica de escoamentos reativos bifásicos turbulentos usando malhas adaptativas}

\begin{abstract}
  Esta é uma proposta de pesquisa na área de Análise Numérica, Computação Científica  e Dinâmica de Fluidos Computacional. O principal objetivo do projeto é apresentar uma ferramenta computacional eficiente para a resolução numérica de Equações Diferenciais Parciais (EDPs) e Equações Diferenciais Ordinárias (EDOs), visando a simulação computacional de problemas da área de Dinâmica de Fluidos. No desenvolvimento desta ferramenta será dada ênfase ao desenvolvimento de técnicas computacionais e ao estudo teórico dos métodos numéricos envolvidos na solução destes problemas, tais como a solução numérica de sistemas lineares, métodos de discretização de domínios, geração de malhas adaptativas, métodos conservativos para a aproximação de EDPs, modelagem da turbulência, solução numérica de sistemas de EDOs, aproximação de dados tabelados, estruturas de dados eficientes e análise dos resultados.
\end{abstract}


\section{Introdução}\label{ss.introducao}

\textcolor{red}{Alterar a introdução para incluir outros problemas de aplicação: Equações de Stokes, ...}

\hspace{1pc} A ciência e a tecnologia dos processos de combustão têm grande impacto nos setores industrial, de transporte e de geração de energia. Dentre os processos de combustão estão aqueles que são altamente nocivos devido a emissão de gases poluentes. Desta maneira, modelos matemáticos são empregados para predizer velocidade, temperatura, concentração de espécies químicas, massa específica e pressão de escoamentos visando minimizar, por exemplo, efeitos nocivos ao meio ambiente. Neste contexto, é crescente o interesse em simulações numéricas de escoamentos reativos com combustíveis renováveis e menos agressivos ao meio ambiente como o etanol ou a biomassa. Devido a característica interdisciplinar dos problemas de aplicação envolvendo escoamentos, cada vez mais o tripé composto por teoria, experimento e simulação computacional, se fortalece.  

O presente projeto tem como principal interesse o desenvolvimento e a análise de métodos numéricos que visam a simulação de problemas da área de combustão. O problema de interesse, trata-se de um escoamento bifásico turbulento reativo, onde o oxidante encontra-se na fase gasosa e o combustível é composto por gotículas líquidas ou por partículas sólidas. 

\textcolor{red}{Outros problemas ...}

\subsection{Objetivos}

O principal objetivo do presente projeto é apresentar uma ferramenta computacional para a resolução numérica de EDPs e EDOs, visando a simulação computacional de escoamentos bifásicos, reativos e turbulentos. Entre os objetivos específicos destacam-se: analisar modelos para problemas físicos envolvendo escoamentos reativos bifásicos turbulentos, desenvolver e implementar métodos numéricos e analisar os resultados obtidos. Para alcançar tais objetivos destacamos as seguintes questões para cada um deles. Primeiramente, vamos utilizar modelos simplificados e aos poucos, incluir mais fenômenos físicos, analisando a influência dos mesmos. Com relação ao desenvolvimento de métodos numéricos, destacamos: a geração de malhas adaptativas dinâmicas, o uso de uma estrutura de dados {\em Tabela de Dispersão} para armazenar e gerenciar os elementos da malha dinâmica, o estudo de diferentes técnicas para a solução dos sistemas lineares, a aproximação de dados tabelados e a implementação de métodos conservativos para aproximar EDPs, além da possibilidade de paralelização destes algoritmos. Durante a análise dos resultados, vamos utilizar técnicas estatísticas e verificar numericamente a ordem de convergência dos métodos estudados. Além disso, vamos comparar os resultados das simulações realizadas com os resultados de simulações e experimentos apresentados na literatura.

\subsection{Relevância da proposta de trabalho}

O presente projeto desempenha um papel interdisciplinar estreitando relações entre a academia e a indústria. Empresas dos setores de energia e transporte (como a Petrobras) buscam soluções para diversos problemas reais como o refino de petróleo e gás, a conversão de biomassa e a produção de biocombustíveis. Além disso, contribuirá na formação de recursos humanos em diferentes níveis (graduação, mestrado e doutorado) nas áreas de Análise Numérica, Computação Científica e Dinâmica de Fluidos Computacional. As metodologias e técnicas que serão apresentadas neste projeto devem contribuir para o progresso do estado da arte e para o tratamento de outros problemas de aplicação das áreas acima citadas.  

As próximas seções estão organizadas da seguinte forma. A Seção~\ref{ss.problema} apresenta uma formulação do problema de interesse. A Seção~\ref{ss.metodologia} apresenta uma proposta de metodologia numérica para a solução dos problemas descritos e cita alguns desafios a serem superados. A Seção~\ref{ss.conclusao} descreve os resultados esperados durante a execução do projeto proposto e um cronograma com as atividades previstas.

\section{Enunciado dos problemas}\label{ss.problema}

\hspace{1pc} {\bf Problema 1}

\hspace{1pc} {\bf Escoamentos reativos bifásicos}

Um dos objetivos do presente projeto é analisar modelos para problemas físicos envolvendo escoamentos reativos, ou seja, escoamentos que envolvam reações químicas. Exemplos de tais sistemas complexos de escoamento são câmaras e motores de combustão interna, condutos tubulares para o transporte de misturas sólidas ou líquidas e transferência de calor com evaporação \cite{VER07}. A velocidade de um escoamento é uma propriedade importante e o caracteriza de diferentes maneiras. Por exemplo, a razão entre a velocidade do escoamento e a velocidade do som (o número de {\em Mach} - $Ma$) determina se os efeitos da {\em compressibilidade} devem ser considerados. Escoamentos compressíveis possuem uma relação termodinâmica conectando densidade, temperatura e pressão. Nos escoamentos incompressíveis, a pressão não possui nenhum efeito termodinâmico. De acordo com Ferziger e Peri\'c \cite{FER03}, se o escoamento possui um número de {\it Mach} baixo, $Ma < 0.3$, ele pode ser considerado incompressível, caso contrário ele é compressível. Se $Ma < 1$, o escoamento é chamado subsônico. Quando $Ma > 1$, o escoamento é supersônico e existe a possibilidade de ocorrerem ondas de choque. Um escoamento com baixa velocidade, no qual cada {\em partícula} do fluido segue uma trajetória suave é dito ser {\em laminar}. O aumento da velocidade pode causar instabilidades que produzem um tipo de escoamento mais randômico e caótico, nesse caso chamado {\em turbulento} \cite{FER03}. Em escoamentos turbulentos existem perturbações nas condições iniciais, nas condições de fronteira e nas propriedades materiais do escoamento. Em geral conceitos e técnicas estatísticas são utilizados na representação de escoamentos turbulentos, tais como médias, flutuações, funções densidade de probabilidade e distribuições \cite{POP01}. A Figura~\ref{FA-flow}(a) apresenta a velocidade de um escoamento laminar e a Figura \ref{FA-flow}(b), a velocidade de um escoamento turbulento. 
\begin{figure}[!ht]
  \begin{center}
    \includegraphics[width=0.35\textwidth]{figuras/laminarA.pdf}
    \vspace{0.5pc}
    \includegraphics[width=0.35\textwidth]{figuras/turbulentoA.pdf}
    \put(-85,-8){{\scriptsize (b)}}
    \put(-255,-8){{\scriptsize (a)}}
  \end{center}
  \vspace{-1.5pc}
  \caption{(a) Escoamento laminar. (b) Escoamento turbulento \cite{CAL12}.}
  \label{FA-flow}
\end{figure}

\subsection{Equações governantes do escoamento bifásico reativo}

No problema de interesse, o combustível é composto por partículas sólidas ou gotículas líquidas e o oxidante está na fase gasosa. O problema é definido em um domínio espacial $\Omega\subseteq\mathbb{R}^n$ e em um intervalo de tempo $[t_i,t_f]$. Vamos considerar um escoamento reativo bifásico com baixo número de {\it Mach} e turbulento \cite{FER03,KUO05,POI05,VER07}. Vamos utilizar a formulação euleriana-lagrangiana que é composta por dois conjuntos de equações. Um para a fase contínua que representará o oxidante, modelada por meio das seguintes equações. 

A equação da conservação da massa,
\begin{equation}
\label{s1.eq1}
\dfrac{\partial \rho}{\partial t} + \nabla\cdot (\rho{\bf u}) = S_m,
\end{equation}
sendo $\rho:=\rho({\bf x},t)$ a massa específica, $\mathbf{u}:=\mathbf{u}({\bf x},t)$ o campo vetorial de velocidade do escoamento, com ${\bf x}\in\Omega$ e $t\in[t_i,t_f]$ e $S_m$ um termo fonte, responsável pelo acoplamento entre as fases.

A equação da quantidade de movimento linear,
\begin{equation}
\label{s1.eq2}
\dfrac{\partial (\rho {\bf u})}{\partial t} + \nabla\cdot(\rho{\bf u}{\bf u}) = -\nabla p_d + \nabla\cdot\left[\mu\left(\nabla {\bf u} + (\nabla {\bf u})^T - \dfrac{2}{3}(\nabla\cdot{\bf u}){\bf I}\right)\right] + {\bf f} + {\bf S}_u,
\end{equation}
sendo $p_d:=p_d({\bf x},t)$ a pressão dinâmica, $\mathbf{f}:={\bf f}({\bf x},t)$ o termo fonte que inclui as forças que agem sobre o sistema, como a força gravitacional, ${\bf I}$ o tensor identidade, ${\bf S}_u$ um termo fonte responsável pelo acoplamento entra as fases e $\mu$ a viscosidade do escoamento.

A equação da conservação de uma propriedade escalar, 
\begin{equation}
\label{s1.eq3}
\dfrac{\partial \rho z}{\partial t} + \nabla\cdot(\rho z{\bf u}) = \nabla\cdot \left(\rho D\nabla z\right) + S_z,
\end{equation}
sendo $D$ o coeficiente de difusão das espécies químicas, $z:=z({\bf x},t)$ a propriedade escalar {\em fração de mistura} e $S_z$ um termo fonte responsável pelo acoplamento entre as fases. A  fração de mistura modela as reações químicas relacionando a temperatura e as espécies químicas envolvidas. Durante a combustão, um combustível (por exemplo, o metano $CH_4$) reage com um oxidante (por exemplo, o ar) para formar os produtos da combustão. O modelo SCRS ({\it Simple Chemical Reacting System}) preocupa-se com a natureza global do processo de combustão. Este modelo assume que a reação química ocorre em um único passo, é instantânea e oxidante e combustível se combinam totalmente para formar os produtos da reação \cite{KUO05,POI05,VER07}. Vale ressaltar que a reação química ocorre na fase gasosa. Dessa forma, o combustível que inicialmente é líquido, evapora e então reage com o oxidante.

A equação do estado termodinâmico,
\begin{equation}
\label{s1.eq4}
p_t=\dfrac{\rho T\mathcal{R}}{W},
\end{equation}
sendo $p_t:=p_t(t)$ a pressão termodinâmica (pressão ambiente), $T:=T({\bf x},t)$ a temperatura do escoamento, $\mathcal{R}$ a constante universal dos gases, $W$ o peso molecular da mistura.  Além disso, o modelo matemático apresentado pelas equações (\ref{s1.eq1})-(\ref{s1.eq4}) requer condições iniciais e de fronteira, que dependem do problema de aplicação.

O segundo conjunto de equações, que representará o combustível conforme \cite{APT09,SAN11,SAN13,SCH11}, modela a fase dispersa por meio das seguintes equações. O transporte de cada partícula $p$, é dado por,
\begin{align}
\label{s1.eq6}
\frac{d \mathbf{x}_p}{d t} & = \mathbf{u}_p,\\
\label{s1.eq7}
m_p\frac{d\mathbf{u}_p}{d t} & = {\bf F}_{c\rightarrow p} + {\bf F}_{ext}, 
\end{align}  
sendo ${\bf x}_p:=\mathbf{x}_p(t)$ a posição do centro de massa da partícula $p$, ${\bf u}_p:=\mathbf{u}_p(t)$ a velocidade da partícula e $m_p$ a massa da partícula. Na equação (\ref{s1.eq7}) ${\bf F}_{c\rightarrow p}$ representa a força exercida pelo fluido (fase contínua) na partícula e ${\bf F}_{ext}$ as forças externas que agem sobre a partícula, tais como a força de arrasto, as causadas por colisões e a gravidade \cite{APT09,CAL12,IRE17,MIN15,RIB09,SCH11}.  O modelo para a fase dispersa ainda inclui uma equação para a evolução da massa, ou seja, a taxa de variação da massa que modela a evaporação da partícula ou gotícula. Além de uma equação para a taxa de variação da temperatura de cada partícula ou gotícula \cite{APT09,SAN13,JIA17,NOH18}.

O acoplamento entre as fases, nos dois sentidos, é feito por meio de uma aproximação suave para o {\it Delta de Dirac} \cite{CAL12,GRI07,MIN15,ROM99}.
Além disso, todas as equações do modelo requerem condições iniciais. A posição inicial das partículas será distribuída uniformemente. A velocidade inicial, será acrescida por uma pequena perturbação. Uma distribuição log-normal será usada para a inicialização do diâmetro das partículas \cite{CAL12,RIB09}.   

Durante a revisão da literatura, serão analisados diferentes modelos para o problema de combustão bifásico, onde o combustível é composto por gotículas de etanol ou por partículas de biomassa. O objetivo desta análise é partir de modelos simplificados e, aos poucos, incluir mais detalhes dos fenômenos físicos envolvidos, inclusive modelos de reação química detalhada. Além disso, vamos estudar modelos estocásticos que incluam os efeitos da turbulência no transporte das partículas \cite{MIN15,POP01}.

\subsection{Método da Projeção}

Em escoamentos com baixo número de {\it Mach} \cite{BEL05,MAD85,PEM98} a pressão pode ser decomposta como a soma da pressão termodinâmica com a pressão dinâmica,
\begin{equation}\label{s1.eq5}
p({\bf x},t) = p_t(t) + p_d({\bf x},t),
\end{equation}
sendo todas as propriedades termodinâmicas independentes de $p_d$.

Dessa forma, para tratar o acoplamento pressão-velocidade da equação (\ref{s1.eq2}) usaremos uma variação do método da Projeção \cite{CHO68, TEM01} para escoamentos com baixo número de {\it Mach}, descrita em \cite{BEL05,CAL12,PEM98}. Isto resultará em uma equação de Laplace para uma variável $\phi$ chamada {\em correção da pressão} que corrige a pressão e o campo de velocidade. A cada passo do tempo, obtemos um campo de velocidade preliminar ${\bf u}^{*}$. No entanto, a seguinte restrição para o divergente do campo de velocidade
\begin{equation}\label{s2.eq2}
  \nabla\cdot{\bf u} = S,
\end{equation}
pode não ser satisfeita. Como o escoamento não é incompressível, o termo $S:=S({\bf x},t)$ pode ser não nulo. Ele é obtido a partir da derivação da equação de estado (\ref{s1.eq4}) combinada à equação de conservação da massa (\ref{s1.eq1}). Para garantir a validade da condição (\ref{s2.eq2}), o método da projeção corrige a pressão $p_d$ e o campo de velocidade ${\bf u}$, fazendo o seguinte,
\begin{align}\label{s2.eq3}
  {\bf u}^{n+1} & = {\bf u^*} + \dfrac{\gamma}{\rho}\nabla \phi,\\
  \label{s2.eq3a}
    p_d^{n+1} & = p_d^n + \phi,
\end{align}
onde $\gamma$ é um parâmetro que depende do método IMEX escolhido em (\ref{s2.eq1}) e $\phi$ é a correção da pressão que é obtida resolvendo-se a equação de Laplace,
\begin{equation}\label{s2.eq4}
 - \nabla \cdot\left(\dfrac{1}{\rho^{n+1}}\nabla \phi\right) = \dfrac{1}{\gamma}\left(\nabla\cdot{\bf u^*} - S^{n+1}\right),
\end{equation}
obtida ao se aplicar o operador $\nabla\cdot$ na equação (\ref{s2.eq3}) \cite{BEL05,CAL12,PEM98}.

\subsection{Modelo para a turbulência}

Em escoamentos turbulentos as propriedades são decompostas em uma parte filtrada (ou média) e uma parte dinâmica (também chamada flutuação) conforme \cite{POI05,POP01,VER07},
\begin{equation}
  q = \tilde{q}+q',
\end{equation}
sendo $\tilde{q}$ a parte filtrada e $q'$ a flutuação. O modelo de turbulência {\it Large Eddy Simulation} (LES) resolve as grandes escalas turbulentas do escoamento e modela as pequenas escalas por meio de modelos de sub-malha (SGS - {\it SubGrid Scale}) após a filtragem das equações (\ref{s1.eq1})-(\ref{s1.eq4}). Vamos utilizar o filtro {\it box} \cite{CAL12,POP01,VER07}. As equações (\ref{s1.eq2}) e (\ref{s1.eq3}) filtradas contém termos adicionais $\mu_t$ (viscosidade turbulenta) e $D_t$ (difusividade turbulenta), que necessitam de modelos \cite{CAL12,FER03,POP01}. Por exemplo, na equação (\ref{s1.eq2}) o modelo de {\it Smagorinsky} para a viscosidade turbulenta é dado por,
\begin{equation}
  \mu_t = (C_s\Delta)^2\mathcal{S},
\end{equation}
sendo $C_s$ uma constante do modelo, $\Delta$ o tamanho do filtro  e $\mathcal{S}$ é módulo da taxa de deformação filtrada \cite{CAL12, POP01}. Já as propriedades como a temperatura e as frações mássicas das espécies químicas são obtidas por meio da função densidade de probabilidade beta ($\beta$-pdf) \cite{CAL12,POI05,VER07}. 

Uma estratégia para simular a turbulência nas condições iniciais do problema é acrescentar uma pequena perturbação ({\it white noise}) em cada uma das propriedades do escoamento \cite{POP01}.


\subsection{Refinamento Adaptativo de Malhas}

A discretização do domínio computacional baseada na técnica AMR ({\it Adaptive Mesh Refinement}) foi proposta por Berger e Colella \cite{BER91,BER84} e trata problemas dinâmicos (que evoluem no tempo), com refinamento adaptativo em regiões de interesse. Com o refinamento localizado temos menos pontos na discretização e assim, menor custo computacional. O refinamento da malha em regiões de interesse se adapta com base em critérios preestabelecidos. A malha refinada em regiões de interesse é construída a partir de uma sequência de malhas aninhadas hierarquicamente. Inicialmente, vamos considerar domínios computacionais bidimensionais. Posteriormente vamos generalizar para domínios tridimensionais. A Figura~\ref{F1-mesh} apresenta uma sequência de construção de uma malha refinada com dois níveis de refinamento e razão de refinamento igual a dois. Um nível de refinamento $l$ é composto por um conjunto de células que possuem as mesmas dimensões, $\Delta x^l$ e $\Delta y^l$. Na Figura~\ref{F1-mesh} temos $\Delta x^l = \Delta y^l$. A razão de refinamento é dada por $\Delta x^{l}/\Delta x^{l+1} = 2$ sendo $l$ um número inteiro não negativo. 
\begin{figure}[!h]
\begin{center}
\subfigure{\begin{tikzpicture}[line cap=round,line join=round,>=triangle 45,x=0.6cm,y=0.6cm]
\draw [line width=1.2pt](-3,1)--(-3,5);
\draw [line width=1.2pt](-3,5)--(3,5);
\draw [line width=1.2pt](3,1)--(3,5);
\draw [line width=1.2pt](-3,1)--(3,1);
\draw (-3,2)--(3,2);
\draw (-3,3)--(3,3);
\draw (-3,4)--(3,4);
\draw (-2,1)--(-2,5);
\draw (-1,1)--(-1,5);
\draw (0,1)--(0,5);
\draw (1,1)--(1,5);
\draw (2,1)--(2,5);
\draw (-0.4,5.5) node {{\scriptsize Nível $l=0$, $\Delta x^0$}};
  \end{tikzpicture}}\hspace{0.5pc}
\subfigure{\begin{tikzpicture}[line cap=round,line join=round,>=triangle 45,x=0.6cm,y=0.6cm]
\draw [line width=1.2pt](-3,1)--(-3,5);
\draw [line width=1.2pt](-3,5)--(3,5);
\draw [line width=1.2pt](3,1)--(3,5);
\draw [line width=1.2pt](-3,1)--(3,1);
\draw (-3,2)--(3,2);
\draw (-3,3)--(3,3);
\draw (-3,4)--(3,4);
\draw (-2,1)--(-2,5);
\draw (-1,1)--(-1,5);
\draw (0,1)--(0,5);
\draw (1,1)--(1,5);
\draw (2,1)--(2,5);
\draw [line width=1.2pt](-3,3)--(-1,3);
\draw [line width=1.2pt](-1,5)--(-1,3);
\draw (-2.5,3)--(-2.5,5);
\draw (-1.5,3)--(-1.5,5);
\draw (-3,4.5)--(-1,4.5);
\draw (-3,3.5)--(-1,3.5);
\draw [line width=1.2pt](1,5)--(1,4);
\draw [line width=1.2pt](1,4)--(0,4);
\draw [line width=1.2pt](0,4)--(0,2);
\draw [line width=1.2pt](0,2)--(3,2);
\draw (0.5,4)--(0.5,2);
\draw (1.5,5)--(1.5,2);
\draw (2.5,5)--(2.5,2);
\draw (0,2.5)--(3,2.5);
\draw (0,3.5)--(3,3.5);
\draw (1,4.5)--(3,4.5);
\draw (0,5.5) node {{\scriptsize Nível $l=1$}};
\draw (-1.5,5.3) node {{\scriptsize $\swarrow$}};
\draw (1.5,5.3) node {{\scriptsize $\searrow$}};
\end{tikzpicture}}\hspace{0.5pc}
\subfigure{\begin{tikzpicture}[line cap=round,line join=round,>=triangle 45,x=0.6cm,y=0.6cm]
\draw [line width=1.2pt](-3,1)--(-3,5);
\draw [line width=1.2pt](-3,5)--(3,5);
\draw [line width=1.2pt](3,1)--(3,5);
\draw [line width=1.2pt](-3,1)--(3,1);
\draw (-3,2)--(3,2);
\draw (-3,3)--(3,3);
\draw (-3,4)--(3,4);
\draw (-2,1)--(-2,5);
\draw (-1,1)--(-1,5);
\draw (0,1)--(0,5);
\draw (1,1)--(1,5);
\draw (2,1)--(2,5);
\draw [line width=1.2pt](-3,3)--(-1,3);
\draw [line width=1.2pt](-1,5)--(-1,3);
\draw (-2.5,3)--(-2.5,5);
\draw (-1.5,3)--(-1.5,5);
\draw (-3,4.5)--(-1,4.5);
\draw (-3,3.5)--(-1,3.5);
\draw [line width=1.2pt](1,5)--(1,4);
\draw [line width=1.2pt](1,4)--(0,4);
\draw [line width=1.2pt](0,4)--(0,2);
\draw [line width=1.2pt](0,2)--(3,2);
\draw (0.5,4)--(0.5,2);
\draw (1.5,5)--(1.5,2);
\draw (2.5,5)--(2.5,2);
\draw (0,2.5)--(3,2.5);
\draw (0,3.5)--(3,3.5);
\draw (1,4.5)--(3,4.5);
\draw [line width=1.2pt](-3,3.5)--(-2,3.5);
\draw [line width=1.2pt](-2,3.5)--(-2,5);
\draw (-3,4.75)--(-2,4.75);
\draw (-3,4.25)--(-2,4.25);
\draw (-3,3.75)--(-2,3.75);
\draw (-2.75,3.5)--(-2.75,5);
\draw (-2.25,3.5)--(-2.25,5);
\draw [line width=1.2pt](1.5,4.5)--(2.5,4.5);
\draw [line width=1.2pt](2.5,4.5)--(2.5,3);
\draw [line width=1.2pt](2.5,3)--(1.5,3);
\draw [line width=1.2pt](1.5,3)--(1.5,2.5);
\draw [line width=1.2pt](1.5,2.5)--(0.5,2.5);
\draw [line width=1.2pt](0.5,2.5)--(0.5,3.5);
\draw [line width=1.2pt](0.5,3.5)--(1.5,3.5);
\draw [line width=1.2pt](1.5,3.5)--(1.5,4.5);
\draw (0.75,2.5)--(0.75,3.5);
\draw (1.25,2.5)--(1.25,3.5);
\draw (1.75,3)--(1.75,4.5);
\draw (2.25,3)--(2.25,4.5);
\draw (1.5,4.25)--(2.5,4.25);
\draw (1.5,3.75)--(2.5,3.75);
\draw (0.5,3.25)--(2.5,3.25);
\draw (0.5,2.75)--(1.5,2.75);
\draw (-0.5,5.5) node {{\scriptsize Nível $l=2$}};
\draw (-2,5.3) node {{\scriptsize $\swarrow$}};
\draw (1.1,4.9) node {$\searrow$};
\end{tikzpicture}}
\end{center}
\vspace{-2pc}
\caption{Uma sequência de malhas aninhadas com dois níveis.}
\label{F1-mesh}
\end{figure}

Inicialmente, no processo de construção, o nível $l=0$ cobre todo o domínio. Caso alguma região necessite de refinamento, as células que a compõem são substituídas por outras, de um nível mais fino. Na Figura~\ref{F1-mesh}, uma célula de um nível $l$ que está em uma região que necessita de refinamento é substituída por quatro células do nível $l+1$. Assim o processo continua, dependendo da  necessidade e da quantidade de níveis de refinamento.


\section{Metodologia numérica}\label{ss.metodologia}

\hspace{1pc} Nesta seção descrevemos brevemente métodos numéricos que serão aplicados para obter uma solução aproximada para o problema descrito anteriormente. Vamos considerar o domínio computacional espacial discretizado pela técnica AMR. Além disso, precisamos discretizar o intervalo de tempo: $t_{n+1}=t_n+\Delta t^{n+1}$, onde $\Delta t^{n+1}$ é o tamanho do passo de tempo em $t^{n+1}$.

\subsection{Integração temporal}

Para a integração temporal utilizaremos métodos IMEX de segunda ordem \cite{ALV13,ASC95,CAL12,HUA19}. A ideia destes métodos consiste em tratar o termo difusivo implicitamente e o termo advectivo explicitamente. Por exemplo, quando aplicado à equação (\ref{s1.eq2}), o tratamento explícito do termo advectivo $\nabla\cdot(\rho {\bf u}{\bf u})$ evita a resolução de um sistema não linear, já que este termo é aproximado considerando os instantes de tempo $t^n$ e $t^{n-1}$. Já o tratamento implícito do termo difusivo $\nabla\cdot\tau$, sendo $\tau = \mu[\nabla{\bf u} + (\nabla{\bf u})^T - 2/3(\nabla\cdot{\bf u}){\bf I}]$ o tensor de tensões viscosas, garante que a escolha do tamanho de passo de integração seja $\Delta t = O(\Delta x)$, dado pela condição de estabilidade do método. Especificamente, um método IMEX de segunda ordem e dois passos no tempo, aplicado à equação (\ref{s1.eq2}) é dado por
\begin{equation}\label{s2.eq1}
\begin{split}
\dfrac{\alpha_2 (\rho {\bf u^*})^{n+1}+\alpha_1(\rho {\bf u})^n + \alpha_0(\rho {\bf u})^{n-1}}{(\Delta t)^{n+1}}  = &  - \beta_1\nabla\cdot(\rho{\bf u}{\bf u})^n - \beta_0\nabla\cdot(\rho{\bf u}{\bf u})^{n-1} + \theta_2[(\nabla\cdot\tau^*)^{n+1} + \\
  & - (\nabla p_d)^{n+1} + {\bf f}^{n+1} + {\bf S}_u^{n+1}] + \theta_1[(\nabla\cdot\tau)^{n} - (\nabla p_d)^n + \\
  & + {\bf f}^n + {\bf S}_u^n] + \theta_0[(\nabla\cdot\tau)^{n-1} - (\nabla p_d)^{n-1} + {\bf f}^{n-1} + {\bf S}_u^{n-1}],
\end{split}
\end{equation}
sendo ${\bf u^*}$ um campo de velocidade preliminar (mais detalhes no próximo tópico) e $\alpha_i$, $\theta_i$ e $\beta_i$, parâmetros que dependem de $\Delta t^{n+1}$ e $\Delta t^n$ e constantes, que definem a combinação de métodos escolhidos. De maneira similar aplicamos o método IMEX às outras equações.

\subsection{Discretização espacial}

Para a discretização espacial utilizaremos o método dos Volumes Finitos. Dessa forma, garantimos as propriedades de conservação nas equações que descrevem o modelo (\ref{s1.eq1})--(\ref{s1.eq4}). A posição das variáveis segue o modelo MAC ({\it marker and cell}: variáveis escalares no centro e as componentes de campos vetoriais nas arestas), veja a Figura~\ref{F4-mesh}(a). Em malhas com refinamento, as células de um nível $l$ podem ser vizinhas de células de um nível diferente. A Figura \ref{F4-mesh} apresenta parte de um domínio computacional com uma interface entre dois níveis de refinamento. Vamos considerar o domínio computacional $\Omega = [a,b]\times[c,d]$ e $\Omega_{ij}^l$ uma célula computacional de um nível $l$ centrada em ${\bf x}^l_{ij}=(a+(i+0.5)\Delta x^l,c+(j+0.5)\Delta y^l)$, com $h_l=\Delta x^l=\Delta y^l$.
\begin{figure}[!h]
\begin{center}
\begin{tikzpicture}[line cap=round,line join=round,>=triangle 45,x=0.8cm,y=0.8cm]
\draw (-3,1)-- (-3,7);
\draw (-3,7)-- (3,7);
\draw (3,7)-- (3,1);
\draw (-3,5)--(3,5);
\draw (-1,1)-- (-1,7);
\draw (-3,1)-- (3,1);
\draw (-3,3)-- (3,3);
\draw (2,1)-- (2,7);
\draw (1,4)-- (3,4);
\draw (1,2)-- (3,2);
\draw (1,6)--(3,6);
\draw (1,7)-- (1,1);
\begin{scriptsize}
\fill (0,4) circle (2.0pt);
\fill (0,2) circle (2.0pt);
\fill (0,6) circle (2.0pt);
\fill (-2,2) circle (2.0pt);
\fill (-2,4) circle (2.0pt);
\fill (-2,6) circle (2.0pt);
\draw [line width=2pt] (-2.9,6) -> (-3.1,6);
\draw [line width=2pt] (-2.9,4) -> (-3.1,4);
\draw [line width=2pt] (-2.9,2) -> (-3.1,2);
\draw [line width=2pt] (-0.9,6) -> (-1.1,6);
\draw [line width=2pt] (-0.9,4) -> (-1.1,4);
\draw [line width=2pt] (-0.9,2) -> (-1.1,2);
\draw [line width=2pt] (-2,6.9) -> (-2,7.1);
\draw [line width=2pt] (-2,4.9) -> (-2,5.1);
\draw [line width=2pt] (-2,2.9) -> (-2,3.1);
\draw [line width=2pt] (-2,0.9) -> (-2,1.1);
\draw [line width=2pt] (0,6.9) -> (0,7.1);
\draw [line width=2pt] (0,4.9) -> (0,5.1);
\draw [line width=2pt] (0,2.9) -> (0,3.1);
\draw [line width=2pt] (0,0.9) -> (0,1.1);
\fill (1.5,5.5) circle (1.0pt);
\fill (1.5,6.5) circle (1.0pt);
\fill (1.5,3.5) circle (1.0pt);
\fill (1.5,4.5) circle (1.0pt);
\fill (2.5,5.5) circle (1.0pt);
\fill (2.5,6.5) circle (1.0pt);
\fill (2.5,4.5) circle (1.0pt);
\fill (2.5,3.5) circle (1.0pt);
\fill (1.5,2.5) circle (1.0pt);
\fill (2.5,2.5) circle (1.0pt);
\fill (1.5,1.5) circle (1.0pt);
\fill (2.5,1.5) circle (1.0pt);
\draw (0.9,5.5) -> (1.1,5.5);
\draw (0.9,6.5) -> (1.1,6.5);
\draw (0.9,4.5) -> (1.1,4.5);
\draw (0.9,3.5) -> (1.1,3.5);
\draw (0.9,2.5) -> (1.1,2.5);
\draw (0.9,1.5) -> (1.1,1.5);
\draw (1.9,6.5) -> (2.1,6.5);
\draw (1.9,5.5) -> (2.1,5.5);
\draw (1.9,4.5) -> (2.1,4.5);
\draw (1.9,3.5) -> (2.1,3.5);
\draw (1.9,2.5) -> (2.1,2.5);
\draw (1.9,1.5) -> (2.1,1.5);
\draw (2.9,5.5) -> (3.1,5.5);
\draw (2.9,6.5) -> (3.1,6.5);
\draw (2.9,4.5) -> (3.1,4.5);
\draw (2.9,3.5) -> (3.1,3.5);
\draw (2.9,2.5) -> (3.1,2.5);
\draw (2.9,1.5) -> (3.1,1.5);
\draw (1.5,5.9) -> (1.5,6.1);
\draw (1.5,6.9) -> (1.5,7.1);
\draw (1.5,4.9) -> (1.5,5.1);
\draw (1.5,3.9) -> (1.5,4.1);
\draw (1.5,2.9) -> (1.5,3.1);
\draw (1.5,1.9) -> (1.5,2.1);
\draw (1.5,0.9) -> (1.5,1.1);
\draw (2.5,6.9) -> (2.5,7.1);
\draw (2.5,5.9) -> (2.5,6.1);
\draw (2.5,4.9) -> (2.5,5.1);
\draw (2.5,3.9) -> (2.5,4.1);
\draw (2.5,2.9) -> (2.5,3.1);
\draw (2.5,1.9) -> (2.5,2.1);
\draw (2.5,0.9) -> (2.5,1.1);
\end{scriptsize}
\draw (0,0.5) node {(a)};
\end{tikzpicture}%}
\hspace{1.5pc}
\begin{tikzpicture}[line cap=round,line join=round,>=triangle 45,x=0.8cm,y=0.8cm]
  \draw (-3,7)--(3,7);
\draw (-3,1)-- (-3,7);
\draw (-3,5)-- (3,5);
\draw (3,7)-- (3,1);
\draw (-1,1)-- (-1,7);
\draw (-3,1)-- (3,1);
\draw (-3,3)-- (3,3);
\draw (2,1)-- (2,7);
\draw (1,4)-- (3,4);
\draw (1,2)-- (3,2);
\draw (1,6)--(3,6);
\draw (1,7)-- (1,1);
\draw [dotted] (0,1)-- (0,7);
\draw [dotted] (0,4)-- (1,4);
\draw [dotted] (0,6)--(1,6);
\draw [dotted] (0,2)-- (1,2);
\begin{scriptsize}
\fill (0,4) circle (2.0pt);
\fill (0,2) circle (2.0pt);
\fill (0,6) circle (2.0pt);
\fill (-2,2) circle (2.0pt);
\fill (-2,6) circle (2.0pt);
\fill (1.5,5.5) circle (1.0pt);
\fill (1.5,6.5) circle (1.0pt);
\fill (1.5,3.5) circle (1.0pt);
\fill (1.5,4.5) circle (1.0pt);
\fill (2.5,4.5) circle (1.0pt);
\fill (2.5,5.5) circle (1.0pt);
\fill (2.5,6.5) circle (1.0pt);
\fill (2.5,3.5) circle (1.0pt);
\fill (1.5,2.5) circle (1.0pt);
\fill (2.5,2.5) circle (1.0pt);
\fill (1.5,1.5) circle (1.0pt);
\fill (2.5,1.5) circle (1.0pt);
\draw (0.5,5.5) circle (1.0pt);
\draw (0.5,6.5) circle (1.0pt);
\draw (0.5,1.5) circle (1pt);
\draw (0.5,2.5) circle (1pt);
\draw (0.5,3.5) circle (1pt);
\draw (0.5,4.5) circle (1pt);
\fill (-2,4) circle (2.0pt);
\draw (2,2) circle (2pt);
\draw (2,4) circle (2pt);
\draw (2,6) circle (2.0pt);
\end{scriptsize}
\draw (0,0.5) node {(b)};
\end{tikzpicture}%}
\hspace{1.5pc}
\begin{tikzpicture}[line cap=round,line join=round,>=triangle 45,x=0.8cm,y=0.8cm]
  \draw (-3,7)--(3,7);
\draw (-3,1)-- (-3,7);
\draw (-3,5)-- (3,5);
\draw (3,7)-- (3,1);
\draw (-1,1)-- (-1,7);
\draw (-3,1)-- (3,1);
\draw (-3,3)-- (3,3);
\draw (2,1)-- (2,7);
\draw (1,4)-- (3,4);
\draw (1,6)--(3,6);
\draw (1,2)-- (3,2);
\draw (1,7)-- (1,1);
\fill [color=gray!15] (-1,3)--(-1,5)--(1,5)--(1,3);
\draw (-1.2,2.8) node {{\scriptsize A}};
\draw (-1.2,5.2) node {{\scriptsize D}};
\draw (1.2,2.8) node {{\scriptsize B}};
\draw (1.2,5.2) node {{\scriptsize C}};
\draw (0.6,3.5) node {{\scriptsize $E_1$}};
\draw (0.6,4) node {{\scriptsize $E_2$}};
\draw (0.6,4.5) node {{\scriptsize $E_3$}};
\draw [line width=1pt] (-1,3)--(-1,5);
\draw [line width=1pt] (-1,3)--(1,3);
\draw [line width=1pt] (-1,5)--(1,5);
\draw [line width=1pt] ((1,3)--(1,5);
\begin{scriptsize}
  \fill (0,4) circle (2.0pt);
  \draw (0,3.6) node {{\scriptsize $(i,j)$}};
  \fill (0,2) circle (2.0pt);
  \draw (0,1.6) node {{\scriptsize $(i,j-1)$}};
  \fill (0,6) circle (2.0pt);
  \draw (0,5.6) node {{\scriptsize $(i,j+1)$}};
  \fill (1.5,3.5) circle (1.0pt);
  \draw (1.5,3.2) node {{\scriptsize $P_1$}};
  \fill (1.5,4.5) circle (1.0pt);
  \draw (1.5,4.2) node {{\scriptsize $P_4$}};
  \fill (2.5,4.5) circle (1.0pt);
  \draw (2.5,4.2) node {{\scriptsize $P_3$}};
  \fill (2.5,3.5) circle (1.0pt);
  \draw (2.5,3.2) node {{\scriptsize $P_2$}};
\draw (2,4) circle (2pt);
\draw (2,3.7) node {{\scriptsize $(i+1,j)$}};
\fill (-2,4) circle (2pt);
\draw (-2,3.6) node {{\scriptsize $(i-1,j)$}};
\draw [line width=2pt] (-1.1,4) -> (-0.9,4);
\draw (0.9,4.5) -> (1.1,4.5);
\draw [line width=2pt] (0.9,4) -> (1.1,4);
\draw [line width=2pt] (0,4.9) -> (0,5.1);
\draw [line width=2pt] (0,2.9) -> (0,3.1);
\draw (0.9,3.5) -> (1.1,3.5);
\end{scriptsize}
\draw (0,0.5) node {(c)};
\end{tikzpicture}
\end{center}
\vspace{-2pc}
 \caption{(a) Localização das variáveis: $\bullet$ escalares e {\bf -} campo vetorial; (b) Células computacionais: $\bullet$ representa uma célula computacional na posição ${\bf x}_{ij}$ e $\circ$, células auxiliares ({\it fantasmas}); e (c) Discretização com correção de fluxos.}
\label{F4-mesh}
\end{figure}

O método dos Volumes Finitos resulta em um conjunto de equações discretas obtido por meio do balanço de fluxos sobre as células computacionais $\Omega^l_{ij}$. Isso significa que na discretização espacial vamos utilizar informações de células vizinhas.  Como um exemplo, vamos considerar a equação de Laplace,  
\begin{equation}\label{s2.eq5}
-\nabla\cdot(\lambda\nabla \phi) = -\dfrac{\partial}{\partial x}\left(\lambda\dfrac{\partial\phi}{\partial x}\right) - \dfrac{\partial}{\partial y}\left(\lambda\dfrac{\partial\phi}{\partial y}\right) = f,
\end{equation}
com alguma condição para as fronteiras, sendo $\lambda$ o coeficiente difusivo e $f$, funções conhecidas. Integrando (\ref{s2.eq5}) sobre cada célula $\Omega^l_{ij}$, usando o teorema de Gauss no lado esquerdo e aproximando o lado direito da equação (\ref{s2.eq5}), obtemos
\begin{equation}\label{s2.eq6}
-\int_{\Omega_{ij}^l}\nabla\cdot(\lambda\nabla \phi)d\Omega  = -\sum_{k=1}^4\int_{\partial\Omega_{ij}^l}\lambda\nabla\phi\cdot{\bf n}dS_k \simeq f_{ij}h_l^2,
\end{equation}
sendo $f_{ij}$ o valor da função $f$ avaliada em ${\bf x}^l_{ij}$ e ${\bf n}$ o vetor unitário normal às arestas de $\Omega_{ij}^l$. As componentes do fluxo ${\bf F} = \lambda\nabla \phi$ devem ser avaliadas em cada uma das arestas. Por exemplo, no volume em destaque na Figura~\ref{F4-mesh}(c), a aproximação do fluxo na aresta $AB$ é dada por
\begin{equation}\label{s2.eq7}
-\int_A^B\lambda\dfrac{\partial\phi}{\partial y}dx \simeq h_l\lambda_{i,j-\frac{1}{2}}\left(\dfrac{\phi_{ij}-\phi_{i,j-1}}{h_l}\right)=\lambda_{i,j-\frac{1}{2}}(\phi_{ij}-\phi_{i,j-1}),
\end{equation}
sendo $\lambda_{i,j-\frac{1}{2}}$ a função $\lambda$ avaliada na posição ${\bf x}^l_{i,j-\frac{1}{2}}$. Assim, se uma célula possui todas as vizinhas no mesmo nível de refinamento, a equação discreta é dada por,
\begin{equation}\label{s2.eq8}
a_w\phi_{i-1,j}+a_e\phi_{i+1,j}+a_c\phi_{ij}+a_s\phi_{i,j-1}+a_n\phi_{i,j+1} = f_{ij},
\end{equation}
sendo $a_s=-(1/h_l^2)\lambda_{i,j-\frac{1}{2}}$, $a_n=-(1/h_l^2)\lambda_{i,j+\frac{1}{2}}$, $a_w=-(1/h_l^2)\lambda_{i-\frac{1}{2},j}$, $a_e=-(1/h_l^2)\lambda_{i+\frac{1}{2},j}$ e $a_c=-(a_n+a_s+a_e+a_w)$. 

Na malha adaptativa, conforme \cite{TRO01}, essa aproximação será feita por meio de interpolações, envolvendo as {\em células fantasmas} \cite{LEV02,TRO01,VER07}, veja Figura~\ref{F4-mesh}(b). Por exemplo, no volume em destaque na Figura~\ref{F4-mesh}(c), a aproximação do fluxo na aresta $BC$ é dada por 
\begin{equation}\label{s2.eq9}
-\int_B^C\lambda\dfrac{\partial\phi}{\partial x}dy \simeq h_l\lambda_{i+\frac{1}{2},j}\left(\dfrac{\phi_{i+1,j}-\phi_{ij}}{h_l}\right)=\lambda_{i+\frac{1}{2},j}(\phi_{i+1,j}-\phi_{ij}).
\end{equation}
Note que $\phi_{i+1,j}$ está localizado em uma célula fantasma. Uma simples aproximação para este valor é dado por,  
\begin{equation}\label{s2.eq10}
\phi_{i+1,j}=\frac{1}{4}(\phi_{P_1}+\phi_{P_2}+\phi_{P_3}+\phi_{P_4}), 
\end{equation}
a qual contém informações que estão no nível $l+1$. Os valores $\phi_{P_i}$ são aproximações de $\phi$ nas posições $P_i$, com $i=1,\ldots,4$, veja Figura~\ref{F4-mesh}(c). Entretanto, para garantir a conservação dos fluxos, na aresta $BC$ é necessário que,
\begin{equation}\label{s2.eq11}
  F_{E_2}=\frac{1}{2}(F_{E_1}+F_{E_3}),
\end{equation}
sendo $F_{E_1}$ e $F_{E_3}$ os fluxos do nível $l+1$ avaliados na aresta $BC$ e $F_{E_2}$ o fluxo do nível $l$ avaliado na aresta $BC$, veja Figura~\ref{F4-mesh}(c). Nossa intenção é garantir a condição de conservação de fluxos (\ref{s2.eq11}).

\subsection{Sistemas lineares}

A cada passo do tempo, temos um sistema de equações lineares para resolver. Em (\ref{s2.eq8}) apresentamos um exemplo de equação deste sistema. Vamos analisar a construção das matrizes associadas a discretização do problema em uma malha adaptativa. As matrizes possuirão uma estrutura esparsa que dependerá da maneira como as células são enumeradas \cite{SAA03}. Vamos seguir o processo de construção  da matriz de coeficientes em uma malha adaptativa apresentado em~\cite{ALV13}, explorando outras formas de enumerar as células da malha adaptativa e reordenar uma enumeração. Vamos analisar duas estratégias para resolver os sistemas lineares. Uma delas, aplicando métodos iterativos, como o {\it Multigrid} algébrico \cite{TRO01}. A segunda estratégia é utilizar métodos de decomposição de domínios que se baseiam na técnica de divisão e conquista \cite{SAA03}. Este último é um passo natural para a paralelização da metodologia.

\subsection{Sistemas de Equações Diferenciais Ordinárias}

A simulação computacional de escoamentos reativos contendo partículas ou gotículas conta com diversos desafios e tem sido objeto de estudo em trabalhos recentes, como \cite{APT09,SAN13,SAN11,MIN15,RIB09}. Estes desafios vão desde os aspectos computacionais da formulação lagrangiana das gotículas ou partículas até os aspectos físicos, como a interação entre elas e a influência da turbulência no transporte das mesmas. Além disso, os sistemas de EDOs que modelam escoamentos envolvendo reações químicas resultam em um sistema de EDOs {\it stiff}, ou seja, {\em rígido}. A {\em rigidez} de um sistema pode estar associada à variação das componentes em escalas de tempo muito distintas. Estes sistemas exigem métodos específicos, como métodos de passo múltiplo, métodos de Runge-Kutta implícitos, métodos baseados na extrapolação de Richardson e métodos de Rosenbrock \cite{BUT03,HUA19}. 
%\vspace{0.5pc}

\subsection{Estruturas de dados}

A nossa proposta de malha com refinamento adaptativo visa armazenar apenas as informações de células computacionais visíveis. Simulações numéricas com malhas que armazenavam células fantasmas demostraram um alto custo computacional, principalmente quando uma malha gerada era composta por muitos blocos com refinamento \cite{CAL12}.  As tabelas de dispersão armazenarão apenas as células visíveis. Essa estrutura de dados  é uma tabela que armazena em cada posição zero ou mais células. Cada célula possui uma chave que é dada pelo seu nível e pelas coordenadas de um de seus vértices. Dessa forma, toda chave é única, ou seja, não existem duas células diferentes com chaves iguais. A chave de uma célula é usada para obter um índice da tabela onde a correspondente célula estará armazenada. Para isso, é utilizada uma {\em função de dispersão}, que recebe a chave de uma célula $c$ e devolve um índice da tabela para $c$. A função de dispersão pode não garantir a injetividade. Assim,  a função pode devolver um mesmo índice na tabela para duas chaves de células diferentes. Neste caso, precisamos de um mecanismo para tratar as colisões \cite{GRI99,MUL12,SZW12}. Escolhemos tratar as colisões com listas. Isso significa que duas células distintas podem ser armazenadas em uma lista que está associada a um único índice da tabela. Temos observado que a função de dispersão via método de Fibonacci apresenta um bom comportamento \cite{KNU97}. Há resultados teóricos que garantem que a tabela de dispersão alcança tempo esperado constante para as operações de inserir, remover e buscar células. Para isso, temos que manter a quantidade de índices da tabela ligeiramente maior que o número total de células que queremos armazenar.
       
\section{Resultados esperados}\label{ss.conclusao}

\hspace{1pc} O presente projeto interdisciplinar possui duração de três anos e conta com a colaboração de quatro pesquisadores: o professor Dr. Alexandre Megiorin Roma (IME-USP), o professor Dr. Álvaro Junio Pereira Franco (INE-UFSC), a professora Dra. Catalina Maria Rua Alvarez (Universidad de Nari\~no, Pasto - Colômbia) e a professora Dra. Priscila Cardoso Calegari (DEC-UFSC).  

Durante este período planejamos desenvolver uma ferramenta computacional utilizando uma tabela de dispersão para armazenar uma malha computacional, métodos conservativos para aproximar as equações diferenciais e métodos iterativos para resolver sistemas lineares esparsos, visando a simulação computacional de escoamentos reativos bifásicos turbulentos. Para tanto, espera-se aplicar a metodologia numérica proposta, incluindo-se a análise de critérios de refinamento de malha, o método de decomposição de domínios, a discretização conservativa das equações diferenciais parciais e o estudo de modelos que envolvam fenômenos físicos complexos, como a turbulência. 

Os principais resultados esperados são:
\begin{enumerate}
  \vspace{-1pc}
\item [(a)] a apresentação de uma ferramenta computacional, devidamente validada, que resolva o problema apresentado;
  \vspace{-0.5pc}
  \item [(b)] a obtenção de resultados nos tópicos propostos que contribuam com o progresso do estado da arte; 
  \vspace{-0.5pc}
\item [(c)] a submissão desses trabalhos em periódicos e congressos das áreas de Análise Numérica, Dinâmica de Fluidos Computacional, Computação Científica e Matemática Aplicada; e 
  \vspace{-0.5pc}
  \item [(d)] o fortalecimento da colaboração entre os pesquisadores envolvidos e a formação de recursos humanos.
  \end{enumerate}

Durante todo o período pretendemos sempre manter a pesquisa atualizada com a literatura. No início do projeto vamos estudar os modelos apresentados no texto. No entanto, pretendemos utilizar modelos mais precisos na representação dos fenômenos físicos de interesse. O desenvolvimento da ferramenta será acompanhado por verificações da ordem de aproximação dos métodos e comparações com resultados encontrados na literatura. Além disso, serão submetidos trabalhos e artigos para congressos e revistas das áreas de Análise Numérica, Computação Científica, Matemática Aplicada e Dinâmica de Fluidos Computacional. 

\begin{footnotesize}
\bibliographystyle{acm}
\bibliography{bibliografia}
\end{footnotesize}


\end{document}
